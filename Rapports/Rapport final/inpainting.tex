\section{Reconstruction de l'image}

Une fois le masque appliqué à l'image, nous sommes ramené à un problème dit d'\emph{inpainting}. Il s'agit alors de trouver une méthode permettant de combler les trous formés par le masquage en respectant la structure de l'image. Il existe de nombreuses méthodes pour cela, nous en avons implémenté quelques unes. 

\subsection{Inpainting par régularisation variationnelle}

\subsubsection{Inpainting avec norme Sobolev}

La résolution du problème d'inpainting repose sur un problème de minimisation d'énergie - ici norme de Sobolev - sous des contraintes de correspondance. Ainsi si on note $Phi$ l'opérateur de masquage de l'image - avec le masque extrait - et $y$ l'image masquée, l'\emph{inpainting} par Sobolev se récrit :
\begin{equation}
f^* = \arg \min E(f) = \|\nabla f \|^2 \text{ sous contraintes } \Phi(f) = y
\end{equation}
Cette fonction à minimiser est régulière et l'on peut utiliser un algorithme de descente de gradient projeté. Si l'on note - $\Omega$ est le masque :
\begin{equation}
(\Pi f)_i = 
\left\{
\begin{array}{lll}
y_i & \text{si} & i\in\Omega \\
f_i & \text{si} & i\notin\Omega
\end{array}
\right.
\end{equation}
l'opérateur de projection orthogonal sur la contrainte $Phi f = y$, alors l'algorithme de descente de gradient projeté consiste à itérer : 
\begin{equation}
f^{(n+1)} = \Pi \left( f^{(n)} + \tau \Delta(f^{(n)})\right)
\end{equation}
où $\Delta = - \nabla^* \circ \nabla$ (l'adjoint du gradient étant -div. Le terme de mise à jour est donc bien le terme classique de gradient dans la descente de gradient - ici c'est le gradient de l'énergie de Sobolev $E(f)$. La condition de convergence de la méthode de descente de gradient projetée est que le pas de mise à jour vérifie : 
\begin{equation}
\tau < \frac{2}{\|\Delta \|}
\end{equation}

Les résultats de l'\emph{inpainting} avec norme de Sobolev sont présenté en figure (A FAIRE). INTERPRETATION.

 
\subsubsection{Inpainting avec norme TV}

On peut aussi remplacer dans le problème d'optimisation précédent la norme de Sobolev par la norme de variation totale (TV). Nous verrons qu'elle a tendance à mieux reconstruire les contours que la norme Sobolev qui floute les images. Pour autant elle n'est pas différentiable en 0, on lui préfère donc sa version régularisée :
\begin{equation}
J_{\epsilon}(f) = \sum_x \sqrt{\|\nabla f(x) \|^2 + \epsilon^2}
\end{equation}
De nouveau on peut réutiliser l'algorithme de descente de gradient projeté, il suffit donc de calculer le gradient de cette nouvelle énergie. Il vaut : 
\begin{equation}
G_{\epsilon} = - \text{div} \left( \frac{(\nabla f)_i}{\sqrt{\| (\nabla f)_i \|^2 + \epsilon^2}}\right)_i
\end{equation}
La contrainte à respecter sur le pas est cette fois-ci :
\begin{equation}
\tau < \frac{\epsilon}{4}
\end{equation}
(RESULTATS + INTERPRETATION).


\subsection{Inpainting par régularisation parcimonieuse}